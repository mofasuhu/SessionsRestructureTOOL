% \iffalse meta-comment
% __________________________________
% The arabic-math package for LuaTeX
%
% Copyright 2010--2018 Khaled Hosny <khaled.hosny@hindawi.com>
%
% This work is under the CC0 license.
%
% Unpacking:
%    luatex arabic-math.dtx
% Documentation:
%    lualatex arabic-math.dtx
%
%<*ignore>
\begingroup
  \def\x{LaTeX2e}%
\expandafter\endgroup
\ifcase 0\ifx\install y1\fi\expandafter
         \ifx\csname processbatchFile\endcsname\relax\else1\fi
         \ifx\fmtname\x\else 1\fi\relax
\else\csname fi\endcsname
%</ignore>
%<*install>
\input docstrip.tex

\keepsilent
\askforoverwritefalse

\let\MetaPrefix\relax

\preamble

See the aforementioned source file(s) for copyright and licensing information.

\endpreamble

\let\MetaPrefix\DoubleperCent

\generate{%
  \usedir{tex/luatex/arabic-math}%
  \file{arabic-math.sty}{\from{arabic-math.dtx}{package}}%
}

\def\MetaPrefix{-- }

\def\luapostamble{%
  \MetaPrefix^^J%
  \MetaPrefix\space End of File `\outFileName'.%
}

\def\currentpostamble{\luapostamble}%

\generate{%
  \usedir{tex/luatex/arabic-math}%
  \file{arabic-math.lua}{\from{arabic-math.dtx}{luamodule}}%
}

\endbatchfile
%</install>
%<*ignore>
\fi
%</ignore>
%<*driver>
\documentclass{ltxdoc}
\usepackage[ltrdocument]{arabic-math}
\usepackage{fontspec}
\setmainfont{STIX Two Text}
\setmonofont[Scale=MatchUppercase]{NoName Fixed}
\setarabicmathfont{XITS Math}
\newfontfamily\arabicfont[Script=Arabic]{Amiri}
\parindent=0pt
\mathdir TLT

\usepackage{etoolbox}
\makeatletter
\gdef\tshortstack{\@ifnextchar[\@tshortstack{\@tshortstack[c]}}
\let\@tshortstack\@shortstack
\patchcmd\@tshortstack\vbox\vtop{}{}
\makeatother

\begin{document}
  \DocInput{\jobname.dtx}
\end{document}
%</driver>
% \fi
% \GetFileInfo{arabic-math.dtx}
%
% \title{The \textsf{arabic-math} package}
% \author{Khaled Hosny \\ \texttt{khaled.hosny@hindawi.com}}
%
% \maketitle
%
% \StopEventually{}
%
% \section{Documentation}
%
% The \textsf{arabic-math} package provides support for typesetting mathematics
% in Arabic mathematical notation.
%
% \subsection{Requirements}
%
% The \textsf{arabic-math} package requires Lua\TeX{} engine (version
% 0.79.\textit{x} or newer), \textsf{unicode-math} package (version
% v0.7c or newer) and \textsf{luatexbase} package (version v0.6 or
% newer).
%
% \subsection{Usage}
%
% The \textsf{arabic-math} package should be loaded with |\usepackage| command.
% No text or math fonts are loaded by default, so they need to be explicitly
% specified. For example:
% \begin{verbatim}
% \documentclass{article}
% \usepackage{arabic-math}
% \setmainfont[Script=Arabic]{Amiri}
% \setarabicmathfont{XITS Math}
% \begin{document}
% نص عربي \(\alef+\beh=\sqrt{\jeem}\).
% \end{document}
% \end{verbatim}
%
% \subsection{Package options}
% \begin{enumerate}
%   \def\ARM#1{{\arabicfont\mathdir TRT $#1$}}
%   \item[|dotlessop|] By default operator names like |\sin| and |\cos| are
%     dotted (\ARM{\arsin}, \ARM{\arcos}), with this option they will be
%     dotless (\ARM{\ardotlesssin}, \ARM{\ardotlesscos}). The dotted version is
%     always available with the |\ar| prefix, and the dotless with |\ardotless|
%     prefix, regardless of the option used, e.g. |\arsin| and |\ardotlesssin|,
%     respectively.
%   \item[|arabicsum|] By default |\sum| will use a mirrored summation symbol
%     (\ARM{\sum}), with this option it will instead use the Arabic summation
%     symbol (\ARM{\arsum}). The Arabic summation symbol is also available as
%     |\arsum|, regardless of the option used.
%   \item[|farsilim|] By default |\lim| will use Arabic limit symbol
%     (\ARM{\arlim}), but with this option Farsi limit symbol will be used
%     (\ARM{\falim}). Both symbols are available as |\arlim| and |\falim|,
%     respectively, regardless of the option used.
%   \item[\smash{\tshortstack[l]{|altalef|\\\hfill |altdal|\\\hfill |altreh|\\ |altnoon|}}]
%     The default forms of Alef, Dal, Reh and Noon are (\ARM{\alef},
%     \ARM{\dal}, \ARM{\reh}, \ARM{\noon}), with this option the alternate
%     forms (\ARM{\alt\alef}, \ARM{\alt\dal}, \ARM{\alt\reh}, \ARM{\alt\noon})
%     will be used. The alternate forms are always available with
%     |\alt{\alef}|, |\alt{\dal}|, |\alt{\reh}|, and |\alt{\noon}|,
%     respectively.
% \end{enumerate}
%
% \subsection{Commands}
% \begin{enumerate}
%   \item[|\string\setarabicmathfont|] Similar to |\setmathfont| from
%     \textsf{unicode-math} package and accepts the same option, and
%     additionally enables extra options needed for Arabic math fonts.
% \end{enumerate}
%
% \subsection{List of Arabic math symbols}
% \DeleteShortVerb{\|}
% \MakeShortVerb{\!}
% \subsubsection{Letters}
% \small
% \begin{tabular}{l c|l c|l c}
% !\alef! & $\alef$ & !\alefi! & $\alefi$ & !\aleft! & $\aleft$ \\
% !\beh! & $\beh$ & !\behi! & $\behi$ & !\beht! & $\beht$       \\
% !\jeem! & $\jeem$ & !\jeemi! & $\jeemi$ & !\jeemt! & $\jeemt$ \\
% !\dal! & $\dal$ & !\dali! & $\dali$ & !\dalt! & $\dalt$ \\
% !\heh! & $\heh$ & !\hehi! & $\hehi$ & !\heht! & $\heht$ \\
% !\waw! & $\waw$ & !\wawi! & $\wawi$ & !\wawt! & $\wawt$ \\
% !\zain! & $\zain$ & !\zaini! & $\zaini$ & !\zaint! & $\zaint$ \\
% !\hah! & $\hah$ & !\hahi! & $\hahi$ & !\haht! & $\haht$ \\
% !\tah! & $\tah$ & !\tahi! & $\tahi$ & !\taht! & $\taht$ \\
% !\yeh! & $\yeh$ & !\yehi! & $\yehi$ & !\yeht! & $\yeht$ \\
% !\kaf! & $\kaf$ & !\kafi! & $\kafi$ & !\kaft! & $\kaft$ \\
% !\lam! & $\lam$ & !\lami! & $\lami$ & !\lamt! & $\lamt$ \\
% !\meem! & $\meem$ & !\meemi! & $\meemi$ & !\meemt! & $\meemt$ \\
% !\noon! & $\noon$ & !\nooni! & $\nooni$ & !\noont! & $\noont$ \\
% !\seen! & $\seen$ & !\seeni! & $\seeni$ & !\seent! & $\seent$ \\
% !\ain! & $\ain$ & !\aini! & $\aini$ & !\aint! & $\aint$ \\
% !\feh! & $\feh$ & !\fehi! & $\fehi$ & !\feht! & $\feht$ \\
% !\sad! & $\sad$ & !\sadi! & $\sadi$ & !\sadt! & $\sadt$ \\
% !\qaf! & $\qaf$ & !\qafi! & $\qafi$ & !\qaft! & $\qaft$ \\
% !\reh! & $\reh$ & !\rehi! & $\rehi$ & !\reht! & $\reht$ \\
% !\sheen! & $\sheen$ & !\sheeni! & $\sheeni$ & !\sheent! & $\sheent$ \\
% !\teh! & $\teh$ & !\tehi! & $\tehi$ & !\teht! & $\teht$ \\
% !\theh! & $\theh$ & !\thehi! & $\thehi$ & !\theht! & $\theht$ \\
% !\khah! & $\khah$ & !\khahi! & $\khahi$ & !\khaht! & $\khaht$ \\
% !\thal! & $\thal$ & !\thali! & $\thali$ & !\thalt! & $\thalt$ \\
% !\dad! & $\dad$ & !\dadi! & $\dadi$ & !\dadt! & $\dadt$ \\
% !\zah! & $\zah$ & !\zahi! & $\zahi$ & !\zaht! & $\zaht$ \\
% !\ghain! & $\ghain$ & !\ghaini! & $\ghaini$ & !\ghaint! & $\ghaint$ \\
% !\dotlessbeh! & $\dotlessbeh$ & !\dotlessbehi! & $\dotlessbehi$ & !\dotlessbeht! \\
% !\dotlessnoon! & $\dotlessnoon$ & !\dotlessnooni! & $\dotlessnooni$ & !\dotlessnoont! & $\dotlessnoont$ \\
% !\dotlessfeh! & $\dotlessfeh$ & !\dotlessfehi! & $\dotlessfehi$ & !\dotlessfeht! & $\dotlessfeht$ \\
% !\dotlessqaf! & $\dotlessqaf$ & !\dotlessqafi! & $\dotlessqafi$ & !\dotlessqaft! & $\dotlessqaft$ \\
% \end{tabular}
%
% \begin{tabular}{l c|l c|l c}
%  !\alefl! & $\alefl$ & !\alefs! & $\alefs$ & !\alefd! & $\alefd$ \\
%  !\behl! & $\behl$ & !\behs! & $\behs$ & !\behd! & $\behd$ \\
%  !\jeeml! & $\jeeml$ & !\jeems! & $\jeems$ & !\jeemd! & $\jeemd$ \\
%  !\dall! & $\dall$ & !\dals! & $\dals$ & !\dald! & $\dald$ \\
%  !\hehl! & $\hehl$ & !\hehs! & $\hehs$ & !\hehd! & $\hehd$ \\
%  !\wawl! & $\wawl$ & !\waws! & $\waws$ & !\wawd! & $\wawd$ \\
%  !\zainl! & $\zainl$ & !\zains! & $\zains$ & !\zaind! & $\zaind$ \\
%  !\hahl! & $\hahl$ & !\hahs! & $\hahs$ & !\hahd! & $\hahd$ \\
%  !\tahl! & $\tahl$ & !\tahs! & $\tahs$ & !\tahd! & $\tahd$ \\
%  !\yehl! & $\yehl$ & !\yehs! & $\yehs$ & !\yehd! & $\yehd$ \\
%  !\kafl! & $\kafl$ & !\kafs! & $\kafs$ & !\kafd! & $\kafd$ \\
%  !\laml! & $\laml$ & !\lams! & $\lams$ & !\lamd! & $\lamd$ \\
%  !\meeml! & $\meeml$ & !\meems! & $\meems$ & !\meemd! & $\meemd$ \\
%  !\noonl! & $\noonl$ & !\noons! & $\noons$ & !\noond! & $\noond$ \\
%  !\seenl! & $\seenl$ & !\seens! & $\seens$ & !\seend! & $\seend$ \\
%  !\ainl! & $\ainl$ & !\ains! & $\ains$ & !\aind! & $\aind$ \\
%  !\fehl! & $\fehl$ & !\fehs! & $\fehs$ & !\fehd! & $\fehd$ \\
%  !\sadl! & $\sadl$ & !\sads! & $\sads$ & !\sadd! & $\sadd$ \\
%  !\qafl! & $\qafl$ & !\qafs! & $\qafs$ & !\qafd! & $\qafd$ \\
%  !\rehl! & $\rehl$ & !\rehs! & $\rehs$ & !\rehd! & $\rehd$ \\
%  !\sheenl! & $\sheenl$ & !\sheens! & $\sheens$ & !\sheend! & $\sheend$ \\
%  !\tehl! & $\tehl$ & !\tehs! & $\tehs$ & !\tehd! & $\tehd$ \\
%  !\thehl! & $\thehl$ & !\thehs! & $\thehs$ & !\thehd! & $\thehd$ \\
%  !\khahl! & $\khahl$ & !\khahs! & $\khahs$ & !\khahd! & $\khahd$ \\
%  !\thall! & $\thall$ & !\thals! & $\thals$ & !\thald! & $\thald$ \\
%  !\dadl! & $\dadl$ & !\dads! & $\dads$ & !\dadd! & $\dadd$ \\
%  !\zahl! & $\zahl$ & !\zahs! & $\zahs$ & !\zahd! & $\zahd$ \\
%  !\ghainl! & $\ghainl$ & !\ghains! & $\ghains$ & !\ghaind! & $\ghaind$ \\
%  !\dotlessbehl! & $\dotlessbehl$ & !\dotlessbehs! & $\dotlessbehs$ & !\dotlessbehd! & $\dotlessbehd$ \\
%  !\dotlessnoonl! & $\dotlessnoonl$ & !\dotlessnoons! & $\dotlessnoons$ & !\dotlessnoond! & $\dotlessnoond$ \\
%  !\dotlessfehl! & $\dotlessfehl$ & !\dotlessfehs! & $\dotlessfehs$ & !\dotlessfehd! & $\dotlessfehd$ \\
%  !\dotlessqafl! & $\dotlessqafl$ & !\dotlessqafs! & $\dotlessqafs$ & !\dotlessqafd! & $\dotlessqafd$ \\
% \end{tabular}
% \normalsize
%
% \subsubsection{Alternate symbols}
% \begin{tabular}{l c}
% !\alt{\alef}! & $\alt\alef$ \\
% !\alt{\dal}!  & $\alt\dal$  \\
% !\alt{\reh}!  & $\alt\reh$  \\
% !\alt{\noon}! & $\alt\noon$ \\
% \end{tabular}
%
% \subsubsection{Numbers}
% \begin{tabular}{c c|c c|c c|c c|c c}
% !0! & $0$ & !1! & $1$ & !2! & $2$ & !3! & $3$ & !4! & $4$ \\
% !5! & $5$ & !6! & $6$ & !7! & $7$ & !8! & $8$ & !9! & $9$ \\
% !.! & $.$ & !,! & $,$ & !\%!& $\%$&     &     &     &     \\
% \end{tabular}
%
% \subsubsection{Operator names}
% \bgroup
% \arabicfont
% \def\ARM#1{{\arabicfont\mathdir TRT $#1$}}
% \begin{tabular}{r l r l r l r l}
% !\sin! & \ARM{\sin} & !\cos! & \ARM{\cos} &
% !\tan! & \ARM{\tan} & !\cot! & \ARM{\cot} \\
% !\sec! & \ARM{\sec} & !\csc! & \ARM{\csc} &
% !\log! & \ARM{\log} & !\ln!  & \ARM{\ln}  \\
% !\lim! & \ARM{\lim} & !\rad! & \ARM{\rad} &
%        &            &        &            \\
% \end{tabular}
% \egroup
%
% \subsubsection{Big operators}
% \begin{tabular}{r l}
% !\sum! & $\arsum$ \\
% !\lim! & $\falim$ \\
% \end{tabular}
%
% \setmathfont[BoldFont={latinmodern-math.otf}]{latinmodern-math.otf}
% \MakeShortVerb{\|}
% \clearpage
% \section{Implementation}
%
%    \begin{macrocode}
%<*package>
%    \end{macrocode}
%
% \subsection{Required packages}
%
%    \begin{macrocode}
\NeedsTeXFormat{LaTeX2e}
\ProvidesPackage{arabic-math}
    [2018/06/06 v0.2 Arabic math support for LuaTeX]
\RequirePackage{luatexbase}[2013/05/11]
\RequireLuaModule{arabic-math}
\RequirePackage{amsmath}
\RequirePackage{unicode-math}[2015/09/24]
\RequirePackage{luacode}
\AtBeginDocument{\directlua{arabicmath.init()}}

\directlua{tex.enableprimitives('', tex.extraprimitives())}
\directlua{luaotfload.log.set_loglevel(-1)}
%    \end{macrocode}
%
% Setup the math fonts.
%
%    \begin{macrocode}

\ExplSyntaxOn
\DeclareDocumentCommand \setarabicmathfont { O{} m O{} }
  {
    \setmathfont {#2}
    [
      RightToLeftAlt,
      StylisticSet=8,
      #1, #3,
    ]

    \setmathfont {#2}
    [
      RightToLeftAlt,
      StylisticSet=8,
      CharacterVariant=1,
      version=alt01
      #1, #3,
    ]
  }
\ExplSyntaxOff

\newfontfeature{RightToLeftAlt}{+rtlm;+rtla}

%    \end{macrocode}
%
% Our module does not do any processing if |\mathdir| is not |TRT| (i.e.
% right-to-left), so set it here. TODO allow documents to switch this on and
% off.
%
%    \begin{macrocode}
\mathdir TRT
%    \end{macrocode}
%
% Make sure equation numbers are on the left.
%
%    \begin{macrocode}
\pagedir TRT
\bodydir TRT
\textdir TRT
\pardir  TRT
%    \end{macrocode}
%
% Use Arabic digits, decimal separator and comma in place of \textsc{ascii}
% ones.
%
%    \begin{macrocode}
\AtBeginDocument{
  \Umathcode`0="7 "0 `٠
  \Umathcode`1="7 "0 `١
  \Umathcode`2="7 "0 `٢
  \Umathcode`3="7 "0 `٣
  \Umathcode`4="7 "0 `٤
  \Umathcode`5="7 "0 `٥
  \Umathcode`6="7 "0 `٦
  \Umathcode`7="7 "0 `٧
  \Umathcode`8="7 "0 `٨
  \Umathcode`9="7 "0 `٩
  \Umathcode`\%="7 "0 `٪
%%\Umathcode`.="0 "0 `٫
  \Umathcode`,="6 "0 `،
}
%    \end{macrocode}
%
% Provide an |\overleftharpoon| command. \textsc{todo} should go to
% \textsf{unicode-math} package.
%
%    \begin{macrocode}
\ExplSyntaxOn
\AtBeginDocument{
  \def\arsum{\arabicmaj}
  \def\falim{\arabichad}
  \iftoggle{arabicsum}{\let\sum\arsum}{}
  \iftoggle{farsilim}{\let\lim\falim}{}
  \iftoggle{altalef}{\let\old@alef=\alef \def\alef{\alt{\old@alef}}} {}
  \iftoggle{altdal} {\let\old@dal=\dal   \def\dal{\alt{\old@dal}}}   {}
  \iftoggle{altreh} {\let\old@reh=\reh   \def\reh{\alt{\old@reh}}}   {}
  \iftoggle{altnoon}{\let\old@noon=\noon \def\noon{\alt{\old@noon}}} {}
}
\ExplSyntaxOff
%    \end{macrocode}
%
% Redefine some text operators. \textsc{todo} http://tex.stackexchange.com/a/199366.
%
%    \begin{macrocode}
\newcommand\DeclareArabicMathOperator[5]{
  \DeclareMathOperator{#2}{\text{#4}}
  \DeclareMathOperator{#3}{\text{#5}}
  \AtBeginDocument{%
    \iftoggle{dotlessop}{\let #1 = #3}{\let #1 = #2}
  }%
}

\DeclareArabicMathOperator {\sin} {\arsin} {\ardotlesssin} {جا}  {حا}
\DeclareArabicMathOperator {\cos} {\arcos} {\ardotlesscos} {جتا} {حٮا}
\DeclareArabicMathOperator {\tan} {\artan} {\ardotlesstan} {ظا}  {طا}
\DeclareArabicMathOperator {\cot} {\arcot} {\ardotlesscot} {ظتا} {طٮا}
\DeclareArabicMathOperator {\sec} {\arsec} {\ardotlesssec} {قا}  {ٯا}
\DeclareArabicMathOperator {\csc} {\arcsc} {\ardotlesscsc} {قتا} {ٯٮا}
\DeclareArabicMathOperator {\log} {\arlog} {\ardotlesslog} {لو}  {لو}
\DeclareArabicMathOperator {\ln}  {\arln}  {\ardotlessln}  {لط}  {لط}

\DeclareMathOperator*{\arlim}{\text{نهـــــا}}
\let\lim=\arlim

\Umathchardef\arrad="7 "0 "0608
\let\rad=\arrad
%    \end{macrocode}
%
% Patch for \textsf{amsmath} to typeset equation numbers in math mode.
%
%    \begin{macrocode}
\let\tagform@old=\tagform@
\def\tagform@#1{\maketag@@@{\ensuremath{(#1)}}}
%    \end{macrocode}
%
% \subsection{Alternate symbols}
%
%    \begin{macrocode}
\newcommand\alt[1]{%
  \mathchoice%
    {\hbox{\mathversion{alt01}$\m@th\displaystyle#1$}}%
    {\hbox{\mathversion{alt01}$\m@th\textstyle#1$}}%
    {\hbox{\mathversion{alt01}$\m@th\scriptstyle#1$}}%
    {\hbox{\mathversion{alt01}$\m@th\scriptscriptstyle#1$}}%
}
%    \end{macrocode}
%
% \subsection{\LaTeX{} package options}
%
%    \begin{macrocode}
\DeclareOption{ltrdocument}{%
    \bodydir TLT
    \pagedir TLT
    \pardir  TLT
    \textdir TLT
}
\newtoggle{dotlessop}
\newtoggle{arabicsum}
\newtoggle{farsilim}
\newtoggle{altalef}
\newtoggle{altdal}
\newtoggle{altreh}
\newtoggle{altnoon}
\togglefalse{dotlessop}
\togglefalse{arabicsum}
\togglefalse{farsilim}
\togglefalse{altalef}
\togglefalse{altdal}
\togglefalse{altreh}
\togglefalse{altnoon}
\DeclareOption{dotlessop}{\toggletrue{dotlessop}}
\DeclareOption{arabicsum}{\toggletrue{arabicsum}}
\DeclareOption{farsilim}{\toggletrue{farsilim}}
\DeclareOption{altalef}{\toggletrue{altalef}}
\DeclareOption{altdal}{\toggletrue{altdal}}
\DeclareOption{altreh}{\toggletrue{altreh}}
\DeclareOption{altnoon}{\toggletrue{altnoon}}
\ProcessOptions\relax
%    \end{macrocode}
%
%    \begin{macrocode}
%</package>
%    \end{macrocode}
%
% \subsection{Lua code}
% The following \textsc{Lua} manipulates the node list to augment Lua\TeX's
% right-to-left math support.
%
%    \begin{macrocode}
%<*luamodule>
%    \end{macrocode}
%
% \subsection{Common variables}
%
%    \begin{macrocode}
arabicmath        = arabicmath or { }

arabicmath.module = {
    name        = "arabic-math",
    version     = 0.2,
    date        = "2018/06/06",
    description = "Arabic math support for LuaTeX",
    author      = "Khaled Hosny",
    copyright   = "Khaled Hosny",
    license     = "CC0",
}

if not modules then modules = { } end modules['arabic-math'] = arabicmath.module

local accentid = node.id("accent")
local mlistid  = node.id("sub_mlist")

local accentattr = luatexbase.new_attribute("accent")
%    \end{macrocode}
%
% \subsection{Utility functions}
%
%    \begin{macrocode}
local function new_dir_node(dir)
    local n
    if node.subtype("dir") then
        n = node.new("whatsit", "dir")
    else
        n = node.new("dir")
    end
    n.dir = dir
    return n
end
%    \end{macrocode}
%
% \subsection{Processing finished math list}
%
% Nodes with `accent' attributes are result from (potentially) extensible
% accents. We flip any successive number of them as they must always be laid
% left-to-right.
%
%    \begin{macrocode}
local function process_math_accents(head)
    if tex.mathdir == "TRT" then
        for n in node.traverse(head) do
            if node.has_attribute(n, accentattr) then
                local accent = n.next
                if accent and node.has_attribute(accent, accentattr) then
                    head = node.insert_before(head, n, new_dir_node("+TLT"))
                    while accent and node.has_attribute(accent, accentattr) do
                        accent = accent.next
                    end
                    head = node.insert_after(head, accent, new_dir_node("-TLT"))
                end
            end
            if n.head then
                n.head = process_math_accents(n.head)
            end
        end
    end
    return head
end
%    \end{macrocode}
%
% \subsection{Preparing math list}
%
% Prepare the math list (full of `noads') before |mlist_to_hlist| is called.
% This used for things where accessing the original math list is required.
%
%    \begin{macrocode}
local function process_sub_mlist(n, func)
     if n and n.id == mlistid then
         func(n.head)
     end
end
%    \end{macrocode}
%
% If this is an accent node, add an `accent' attribute to the accents. We use
% this attribute after |mlist_to_hlist| is done to flip multi-glyph accents as
% they should always be laid out left-to-right.
%
%    \begin{macrocode}
function prepare_math_accents(head)
    if tex.mathdir == "TRT" then
        for n in node.traverse(head) do
            process_sub_mlist(n.nucleus, prepare_math_accents)
            process_sub_mlist(n.sup, prepare_math_accents)
            process_sub_mlist(n.sub, prepare_math_accents)
            process_sub_mlist(n.degree, prepare_math_accents)
            process_sub_mlist(n.num, prepare_math_accents)
            process_sub_mlist(n.denom, prepare_math_accents)
            if n.id == accentid then
                if n.accent then
                    node.set_attribute(n.accent, accentattr, 1)
                end
                if n.bot_accent then
                    node.set_attribute(n.bot_accent, accentattr, 1)
                end
            end
        end
    end
    return head
end
%    \end{macrocode}
%
%    \begin{macrocode}
local function handle_math(head, ...)
    head = prepare_math_accents(head)
    head = node.mlist_to_hlist(head, ...)
    head = process_math_accents(head)
    return head
end
%    \end{macrocode}
%
%    \begin{macrocode}
local function registercallback()
    luatexbase.add_to_callback("mlist_to_hlist", handle_math, "arabic-math (math)", 1)
end
%    \end{macrocode}
%
%    \begin{macrocode}
local letters = {
    "alef", "beh", "jeem", "dal", "heh", "waw", "zain", "hah", "tah", "yeh",
    "kaf", "lam", "meem", "noon", "seen", "ain", "feh", "sad", "qaf", "reh",
    "sheen", "teh", "theh", "khah", "thal", "dad", "zah", "ghain",
    "dotlessbeh", "dotlessnoon", "dotlessfeh", "dotlessqaf",
}

local ranges = {
    [""] =0x1EE00, -- Isolated
    ["i"]=0x1EE20, -- Initial
    ["t"]=0x1EE40, -- Tailed
    ["s"]=0x1EE60, -- Stretched
    ["l"]=0x1EE80, -- Looped
    ["d"]=0x1EEA0, -- Double-struck
}

local function definechars()
    for suffix, start in next, ranges do
        for index, name in next, letters do
            local cmd = string.format('\\Umathchardef\\%s%s="7 "0 "%X', name, suffix, start + index - 1)
            tex.print(cmd)
        end
    end
end

function arabicmath.init()
    definechars()
    registercallback()
end
%    \end{macrocode}
%
%    \begin{macrocode}
%</luamodule>
%    \end{macrocode}
%
% \Finale
\endinput
